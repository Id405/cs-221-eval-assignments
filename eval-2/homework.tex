\documentclass{article}

\usepackage[T1]{fontenc}
\usepackage[default]{gillius}

\usepackage{amsmath}
\usepackage{amssymb}
\usepackage{amsfonts}
\usepackage{amsthm}

\usepackage{hyperref}

\renewcommand{\thesection}{\arabic{section}.}
\renewcommand{\thesubsection}{\alph{subsection}.}

\title{CS 221 Numbers Evaluation Assignment}
\author{Lily Larsen}

\begin{document}
\maketitle
\newpage

\section{}
\href{https://github.com/Id405/cs-221-eval-assignments/blob/main/eval-2/1.c}{Click here}
\newpage

\section{}
\href{https://github.com/Id405/cs-221-eval-assignments/blob/main/eval-2/2.c}{Click here}
\newpage

\section{}
\href{https://github.com/Id405/cs-221-eval-assignments/blob/main/eval-2/3.c}{Click here}
\newpage

\section{}
\href{https://github.com/Id405/cs-221-eval-assignments/blob/main/eval-2/4.c}{Click here}
\newpage

\section{}
\subsection{}
\begin{align*}
    \text{min}&: 0 \\
    \text{max}&: 2^{43} - 1
\end{align*}
\subsection{}
\begin{align*}
    \text{min}&: -2^{42} \\
    \text{max}&: 2^{42} - 1
\end{align*}
\subsection{}
\begin{align*}
    \text{min}&: 0 \\
    \text{max}&: 2^{11} - 1
\end{align*}
\subsection{}
\begin{align*}
    \text{min}&: -2^{10} \\
    \text{max}&: 2^{10} - 1
\end{align*}
\newpage

\section{}
\subsection{}
The largest tiny floating point would have a sign of \(0\), an exponent of \(7\) (or \(1110_2 - 7\)) and a mantissa of \(1.875\) (or \(1.111_2\)), and so would equal \((-1)^{0} \cdot 1.875 \cdot 2^{7} = 240\). The second-largest tiny floating point would have the same sign and exponent and a mantissa of \(1.75\) (or \(1.111_2\)) which would equal \((-1)^0 \cdot 1.75 \cdot 2^{7} = 224\). So, the difference between the largest non-infinite number representable with tiny floating points and the second largest would be \(240-224=16\).
\subsection{}
The smallest positive tiny floating point would have a sign of \(0\), an exponent of \(-6\) (or \(0000_2\)) and a mantissa of \(0.125\) (or \(0.001_2\) due to denormalized encoding), and so would equal \((-1)^{0} \cdot 0.125 \cdot 2^{-6} \approx 0.001953125 = 0.000000001_2\). The second smallest would have the same sign and exponent and a mantissa of 0.25 (or \(0.010_2\)) which would equal \((-1)^{0} \cdot 0.25 \cdot 2^{-6} \approx 0.00390625\) or exactly \(0.00000001_2\). So the difference between the two would approximately be \( 0.00390625 - 0.001953125 \approx 0.001953125 = 0.000000001_2 \).
\end{document}